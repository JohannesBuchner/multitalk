\documentclass[12pt,a4paper,twoside]{article}
\usepackage{graphicx,fancyhdr}

\setlength{\parindent}{0cm}
\setlength{\parskip}{2ex plus1ex minus 0.5ex}

\addtolength{\evensidemargin}{-2.5cm}
\addtolength{\oddsidemargin}{-0.5cm}
\addtolength{\textwidth}{3cm}

\addtolength{\headheight}{0.2cm}
\addtolength{\topmargin}{-2.5cm}
\addtolength{\textheight}{2.5cm}

% \newcommand{\source}[1]{\textbf{\verb^#1^}}}
\renewcommand{\_}{\texttt{\symbol{95}}}
\addtolength{\fboxsep}{0.1cm}
\newcommand{\param}[1]{\textit{\textrm{\textmd{#1}}}}
\newcommand{\codebar}{\rule{\textwidth}{0.3mm}}

\newlength{\codelen}
\newcommand{\code}[1]
{\begin{center}\fbox{\parbox{16cm}{\texttt{#1}}}\end{center}}

\newcommand{\todo}{\textbf{TODO}}
\newcommand{\mission}[1]{\item[#1:]}
% \newcommand{\mission}[1]{\texttt{#1}\hspace{3mm}}

\fancyhead{}
\fancyhead[RO,LE]{\thepage}
\fancyhead[LO,RE]{Multitalk}
\fancyfoot{}
\pagestyle{fancy}
% \pagestyle{empty}

\newenvironment{bulletlist}
{
	\begin{itemize}
	\addtolength{\itemsep}{-1mm}
	% \setlength{\itemsep}{0ex}
	\setlength{\parsep}{0ex}
}
{
	\end{itemize}
}

\newenvironment{alphalist}
{
	\begin{enumerate}
	\setlength{\itemsep}{0ex}
	\setlength{\parsep}{0ex}
	\renewcommand{\labelenumi}{(\alph{enumi})}
}
{
	\end{enumerate}
}

\newenvironment{numericlist}
{
	\begin{enumerate}
	\addtolength{\itemsep}{-1mm}
	% \setlength{\itemsep}{0ex}
	\setlength{\parsep}{0ex}
}
{
	\end{enumerate}
}

\begin{document}

\centerline{\textbf{\LARGE Multitalk}}
\vspace{0.5cm}
\centerline{Version 1.4, 1st September 2008}
\centerline{David Ingram (\texttt{dmi1000@cam.ac.uk})}

{ \parskip -3mm plus 0.1mm \tableofcontents }
\newpage

Copyright (c) David Ingram 2006-8

This program is free software; you can redistribute it and/or modify it
under the terms of the GNU General Public License (version 2) as
published by the Free Software Foundation. See the file COPYING for
more details.

\section{Why another presentation program?}

Multitalk is a new kind of presentation program.

Other presentation programs force you to arrange your talk in a fixed order,
and then to step through it in sequence -- often while reading out the slides to
the audience in a tedious manner! This means that in those programs:

\begin{bulletlist}
\item There's no way to change the direction of your talk at presentation-time
  based on audience feedback.
\item You can't have ``optional'' parts which you expand upon for longer, more
  detailed talks and present at an overview level otherwise, except
  by skipping slides.
\item There's no easy way to show central concepts or illustrate one's
  progress through the structure of the talk, except by regularly
  reinserting copies of the contents page.
\item The speaker isn't actively involved in steering through and explaining
  the material, which can lead to poor communication.
\item Updating multiple talks with shared material is difficult.
\item You cannot view all of your talks in the same space.
\item Presentation concerns are emphasised over content because slides are
  initially designed in a GUI rather than as plain text.
\item Often the same template is used for all slides in a talk; putting
  the same abstract-but-vaguely-hitech-looking coloured pattern in the
  corner of each slide doesn't really add much to the talk.
\end{bulletlist}

\section{What does Multitalk do better?}

\begin{bulletlist}
\item Slides are laid out for display in a \textit{two}-dimensional space which
  you can choose to navigate in real time in any direction.
\item Slide text is written in a markup language using a normal text editor.
\item Image placement is still controlled in a WYSIWYG manner using the program
  itself (which is quicker than pure markup language spacing commands).
\item Slides can be any size and automatically resize to fit their contents;
  you can view multiple small ones at once or part of very large ones, so
  logical units don't have to be stretched to screen size.
\item You can select two slides from anywhere in the talk and view them
  side-by-side.
\item You can zoom out to rearrange slides or get an overview of the talk.
\item Each slide acts like a folding text editor, with definable blocks
  which can be collapsed and expanded during the talk. You can fold
  anything, including images.
\item You can define hyperlinks between different slides, which are spatially
  animated.
\item You can define your own style templates to override all visual aspects,
  and each slide can independently use any style to highlight topics as
  required (you can have multiple ``master slides'' in other terminology).
\end{bulletlist}

\section{Requirements for Linux}

\begin{bulletlist}
\item Linux (and hence gcc, make, X11)
\item \verb=SDL=, \verb=SDL_image=, \verb=SDL_ttf=, \verb=SDL_gfx=
\end{bulletlist}

These versions of the SDL libraries are known to work
(more recent ones should be OK too of course):
\begin{bulletlist}
\item \verb=SDL-1.2.7=
\item \verb=SDL_image-1.2.3=
\item \verb=SDL_ttf-2.0.6=
\item \verb=SDL_gfx-2.0.13=
\end{bulletlist}

To check which of these you already have, try

\begin{verbatim}
  /sbin/ldconfig -v | grep -i sdl
\end{verbatim}

You can get all these in source form from the SDL webpage:
\verb^http://www.libsdl.org/^ (click ``Libraries'' for the add-on libraries).
They all build with the usual

\begin{verbatim}
  ./configure
  make
  make install
\end{verbatim}

Note that \verb^SDL_ttf^ further relies on \verb^libfreetype^;
however most Linux distributions already come with this.

Some Linux distributions come with \verb^SDL^, \verb^SDL_image^ and
\verb^SDL_ttf^ preinstalled but not \verb^SDL_gfx^, in which case you
can install just the latter library yourself.

If you do not have root access, replace the first step with

\begin{verbatim}
  ./configure --prefix=$HOME
\end{verbatim}

This will install the libraries you need in \verb^$HOME/lib^ and the
include files in \verb^$HOME/include^. Multitalk's build process
will find them in these directories. If you install into your
home directory in this way you will also need to set your
loadable library environment variable before running the program:

\begin{verbatim}
  export LD_LIBRARY_CONFIG=$HOME/lib:$LD_LIBRARY_CONFIG
\end{verbatim}

\section{Additional requirements for embedded LaTeX}

\begin{bulletlist}
\item latex
\item dvips
\item convert (part of ImageMagick)
\end{bulletlist}

\section{Requirements for Mac OS X}

\begin{bulletlist}
\item Mac OS X
\item X11
\item SDL, SDL\_image, SDL\_ttf, SDL\_gfx
\end{bulletlist}

You can get the libraries you need from DarwinPorts.

\section{Known issues on Mac OS X}

The installation target (\verb^make install^) requires GNU options to
the \verb^cp^ command which are not available -- you are advised to copy
the built files to a suitable location by hand after running \verb^make^.
Look at the \verb^Makefile^ to see what needs to be copied.

The memory use display feature does not work, because this is coded to use the
Linux /proc filesystem. This could probably be enabled for OS X later by
changing it to use \verb=getrusage()= on this system, but currently the memory
display should not be activated.

Running SDL on X11 on Mac OS the mouse pointer appears to be constrained to the
screen, even when in full screen mode and with an invisible pointer. This is a
slight annoyance if you want to scroll a long distance in one go; it is
necessary to move the mouse back to the centre occasionally instead of
completing the scroll with a single mouse drag.

\section{Note on RPMs}

Some distributions with this package format don't include all the files you
need to compile against the libraries unless you also install a separate set of
``-devel'' packages (e.g. SDL-devel, etc). Symptoms of this might be that you
don't have the \verb=sdl-config= program or necessary header files. If you still
can't get your distribution's packages to work you may find that removing the
RPMs and building the libraries from source is actually easier.

\section{Installation}

First make sure you have all the prerequisites installed, as
described in ``Requirements'' above. Then, if you have root access,
run the following commands:

\begin{verbatim}
  tar xzvf multitalk-1.0.tar.gz
  cd multitalk
  make
  su -c make install
\end{verbatim}

The default installation puts things in \verb^/usr/local^. You can tweak the
Makefile if you want to change that.

If you do not have root access, replace the final step with:

\begin{verbatim}
  make userinstall
\end{verbatim}

If you install Multitalk in a non-standard location and/or it has
trouble locating resources (fonts, styles and graphics) you may need to
set your \verb^MULTITALK_DIR^ environment variable (see the
``File locations'' section below).

\section{How it works}

You create a file with the \verb=.talk= extension, say \verb=foo.talk=. When
Multitalk runs it creates a file called \verb=foo.graph=. The \verb=.talk= file
contains the text for your talk and some markup for formatting
purposes, whereas the \verb=.graph= file contains the coordinates of each
slide in 2D space. The former is always editable by you, but the latter
should not generally be edited by hand because it is maintained by
Multitalk.

\section{Usage}

\begin{verbatim}
multitalk [options...] <talk-name>
\end{verbatim}

\textbf{Options:}

\begin{tabular}{ll}
\verb^-fs^      & full-screen\\
\verb^-win^     & window\\
\verb^-export^  & export HTML\\
\verb^-nowatch^ & don't watch talk file for changes\\
\verb^-reverse^ & reverse mouse scroll direction\\
\verb^-force^   & regenerate latex segments even if text hasn't changed\\
         & (useful if style parameters such as text or background\\
         & colour have been modified for slides containing latex)\\
\verb^-version^ & display the Multitalk version number and then exit\\
\verb^-displaysize=^\textit{width}\verb^x^\textit{height}
         & set window/screen size in pixels\\
\end{tabular}

The talk-name is the part of the filename without the \verb=.talk=
extension (e.g. \verb=foo= for \verb=foo.talk=).

If you don't specify either \verb^-fs^ or \verb^-win^,
the default is to open fullscreen if
the maximum resolution of the display device is the required size,
and otherwise to open in a window.

\section{Examples}

A sample talk called \verb=about.talk= is provided in the examples directory.
This is suitable for explaining Multitalk itself to an audience.
You may want to merge these slides with your own talks. To view
the sample talk, run

\begin{verbatim}
  multitalk examples/about.talk
\end{verbatim}
  
(Specify the full path to the \verb=about.talk= file).

\section{Mouse controls}

Drag the mouse with the left mouse button to scroll the window. The
window scrolls in the direction you move the mouse. If you prefer to
``pull'' the canvas instead, start Multitalk with the \verb^-reverse^ command
line option to invert the mouse drag directions.

Drag the mouse with the right mouse button to zoom out and scroll
(when you release the right mouse button the window zooms back in,
centred on the centre of the screen).
The zoom is by a factor of 3 on both X and Y axes.

Click with the left mouse button on a line with a fold icon (a square
containing a small triangle) to unfold/fold.

Click with the left mouse button on a hyperlink to follow it.

You can also return from a hyperlink by holding down the left mouse
button, then pressing the right one. This is useful if you are
controlling the entire presentation by wireless mouse and can't
reach the keyboard in order to press the Return key (see below).

Click with the left mouse button on the title bar of a card in a slide
stack (see below) to advance to the next card in the stack.

Hold down the \textbf{Shift} key and drag with the left mouse button over
any slide or image to move it. The coordinates of moved slides are
automatically saved for you when you quit the program.

In zoomed out mode, you can also drag slide titlebars with the left
mouse button without requiring the shift key (a bit like rearranging
windows with their titlebar on your desktop).

Hold down the \textbf{Ctrl} key and move the mouse pointer over slides
in order to select a group of slides. These can then be dragged to a
new position as a group. Selected slides are shown with red corners.
Press and release the \textbf{Ctrl} key without selecting any more
slides in order to clear a group selection.

\section{Slide gravity}

When you release the mouse button after dragging, if there is a slide
somewhere in approximately the middle of the screen, the display will
automatically scroll in order to centre it (on the assumption that this
is the slide you want to talk about). This means that you just have to
push the mouse to roughly the right position and don't waste time
positioning exactly when you should be looking at the audience.

Occasionally you don't want slide gravity, for example when you are
trying to view multiple smaller slides at once. If the \textbf{Alt} key
is held down when the mouse button is released then the slide gravity
effect will not take place for that action. You can also toggle slide
gravity permanently off or on by pressing the \textbf{G} key.

\section{Zoom levels}

Normal operation is zoom level 0. Use of the spacebar or right mouse button
switches to zoom level 1. Shift-space enters zoom level 2. Zoom level 1
reduces slides by the factor 3x3, whereas zoom level 2 uses 9x9.
Slide features such as hyperlinks, card stack rotation and folding
sections can be operated at zoom levels 0 or 1, but not 2.

Repositioning
slides at zoom level 0 must be accomplished by holding down the shift
key, then dragging with the left mouse button. At zoom level 1, there
is the additional option of dragging the slide titlebars without
needing the shift key. At zoom level 2, slides can be moved by dragging
them anywhere - not just in the titlebar.
Selecting a group of slides and repositioning them together may be
done at any of the three zoom levels.

\section{Keyboard controls}

Press the \textbf{N} key to switch on a translucent navigation
``radar'' window, which appears in the bottom-right corner of the screen.

The four \textbf{cursor} keys can be used to move between slides, as an
alternative to dragging with the mouse. When you press one of these
keys, Multitalk looks in that general direction from the current
centre of the screen and decides which slide is closest. It then
scrolls across so that this slide is centred on the screen.

Press the \textbf{Space} bar to switch to constant zoom mode (press space
again to switch zoom off). This is mainly useful for rearranging
your slides when editing a talk.

Press \textbf{Shift-Space} to switch to zoom level 2 (extra zoom).

Press the \textbf{TAB} key to toggle fullscreen mode.

Press \textbf{Return} to go back to the launch point from the last hyperlink.

Press the `\textbf{.}' (period) key whilst the mouse is over a slide stack to
advance to the next card in the stack.

Press the `\textbf{,}' (comma) key whilst the mouse is over a slide stack to
return to the previous card in the stack.

Press the \textbf{A} key to toggle align-to-grid mode for slides and images.

Press the \textbf{E} key to switch between two examine modes.
This changes what happens when you hover the mouse pointer over a slide
at zoom level 2. The default mode shows the slide title at the
top of the screen; the alternate mode shows a small version of
the entire slide at the bottom-right of the screen.

Press the \textbf{G} key to toggle slide gravity.

Press the \textbf{M} key to switch on an on-screen display of Multitalk's
current memory usage.

Press the \textbf{P} key to toggle a highly visible mouse pointer
(a giant arrow which points in from the edge of the screen).
This can be used as an alternative to a laser pointer.

Press the \textbf{R} key to ask Multitalk to re-read the current `.talk' file.
This is useful when you are simultaneously editing the file and
viewing the results in window mode on a large display.
Note: it is only necessary to do this manually if you have started
Multitalk with the \verb^-nowatch^ command line argument; otherwise
Multitalk checks the file for changes automatically twice a second.

Press the \textbf{Backspace} key to ``pin'' a slide to the top-right corner of
the screen. At zoom levels 1 or 2 the
slide under the mouse pointer will be pinned, whereas at zoom level 0
the slide at the centre of the screen (if any) is pinned. If there
is already a pinned slide, pressing \textbf{Backspace} again unpins it.
Pinned slides are reduced to 50\% in size and made slightly transparent;
they are frozen so you cannot manipulate features such as folding
sections, hyperlinks or stack rotation within them.
Be aware that due to the reduction in scale, pinned slides are
difficult for your audience to read!

Press the \textbf{/} (forward slash) key for the maximise (single
or dual slide view) mode. This operates on the slide at the
centre of the screen, or under the mouse pointer if you are at
zoom levels 1 or 2.
The chosen slide is maximised so that
it fills as much of the screen as possible (without distorting
its aspect ratio). This will generally enlarge the slide, although
it can also be used to reduce a slide to fit the screen, if it is too
big (of course, it is not good practice to design slides larger
than the screen, but this might happen if you are importing a big
diagram, or if unfolded sections make a slide overflow).
The maximised view is static: you cannot scroll around when maximised
nor activate slide features such as hyperlinks, folding sections
or stack rotation.
To exit maximised view, press any key or mouse button.

If the maximise key is pressed whilst a slide is pinned, it enters
dual slide view. In this case both the
chosen slide and the pinned slide are shown side by side (and no
others). This is intended for situations where you need to compare
two slides from possibly distant parts of the talk. Dual slide view
is a static display, like single slide view.
The program decides intelligently whether to split the screen
horizontally or vertically, and what proportion of the space to
allocate to each slide, so that they are as large as possible and
displayed at the same zoom factor.
Note that you cannot compare two different cards from the same stack
with dual slide view, because the current card from each slide is
always shown.

Press \textbf{Escape} to quit.

\section{Markup language}

\begin{bulletlist}
\item The order in which you define slides has no significance
  (you are going to move them around in 2D with Multitalk anyway)
\item Whitespace is taken literally (including newlines)
\item Lines are not wrapped for you
\item Tabs expand to three spaces
\end{bulletlist}

\subsection*{Whole line instructions}

\begin{tabular}{llp{11cm}}
\verb^@ Slide Title^ & -- & introduces a new slide (ending the previous one)
	and specifies its title\\
\verb^@!path/to/image.png^ & -- & create an ``image slide'', the entire
	contents of which will be the image you specify (no title bar or text
	body). Any content lines are ignored until the next \verb^@^ line.\\
\verb^!path/to/image.png^ & -- & display an image within the current slide.
	This kind of image cannot be dragged out of the slide's area.\\
\verb^[^             & -- & start of folding section\\
\verb^]^             & -- & end of folding section\\
\verb^:Slide Title^  & -- & turns the \textit{previous line} into a hyperlink,
	the target of which is the named slide. If the previous line displays
	an image, the image becomes the hyperlink, instead of the most recent
	line of text\\
\verb^:Slide Title.n^  & -- & a hyperlink as above, which also switches to
	card number \verb^n^ within the target stack\\
\verb^!Style^        & -- & sets the current slide's style (may be omitted for defaults)\\
\verb^!n^   & -- & indicates start of a block to appear only on card number n (1-9)\\
\verb^!pqr^ & -- & start of a block to appear on cards p, q, r, etc (two or more)\\
\verb^!!^   & -- & switch back to displaying the same thing on all cards\\
\verb^\^    & -- & beginning or end of LaTeX section\\
\verb^\)^   & -- & beginning of centred LaTeX section\\
\verb^%space n^      & -- & vertical space of n pixels\\
\verb^--^   & -- & horizontal rule\\
\verb^@canvascolour=colour^ & -- & sets global canvas colour
	(the area not covered by any slides)\\
\verb^@designsize=^\textit{width}\verb^x^\textit{height} & -- & the resolution this presentation has
	been designed to be displayed at\\
\end{tabular}

\subsection*{Beginning of line instructions}

\begin{tabular}{llp{12cm}}
\verb^* line text^   & -- & bullet point\\
\verb^** line text^  & -- & second level bullet point\\
\verb^*** line text^ & -- & third level bullet point\\
\verb/^ line text/   & -- & indent this line the same distance as the
	previous one (useful for continuing a bullet point onto another line
	without another bullet)\\
\verb^)line text^    & -- & centre line\\
\verb^# line text^   & -- & comment line\\
\verb^> line text^   & -- & heading line\\
\verb^>) line text^   & -- & heading line (centred)\\
\end{tabular}

\subsection*{Anywhere in text}

\begin{tabular}{llp{13cm}}
\verb^$foo bar$^  & -- & fixed width font (the string may span multiple lines)\\
\verb^*foo bar*^  & -- & bold\\
\verb^/foo bar/^  & -- & italic\\
\verb^%colour.^	& -- & change text colour\\
\verb^%.^			& -- & revert to the default text colour\\
\verb^\X^			& -- & a literal escape (particularly useful X's are
	\verb^*, /, $, \, %,^ and less commonly \verb$@, !, ), ^, :, [, ]$)\\
\end{tabular}

\section{Slide stacks}

Slide stacks allow you to step through a sequence of changes
(new bullet points appearing, graphs animating etc) within a slide.
These are shown visually as a small deck of cards on the canvas.
Click in the slide titlebar to advance to the next card in the stack,
or press comma/period with the mouse over the slide to go back/forward.
In your \verb=.talk= file you introduce a section of the slide to appear only
on certain cards in the stack with one of the following types of line:

\begin{verbatim}
!n   - indicates start of a block to appear only on card number n (1-9)
!pqr - start of a block to appear on cards p, q, r, etc (may be any number)
!!   - switch back to displaying the same thing on all cards
\end{verbatim}

Each of these directives applies to all subsequent lines until you
specify otherwise or until the end of the current slide.
Card 1 is the initial card. You may refer to cards upto number 9.
The stack height will be the same as the maximum card number used.
Blocks are \textit{not} stretched to fill the same space -- you must pad with
blank lines if sections on different cards are of non-equal height and
you want to line up repeated content lower down the slide.

\section{Styles}

Styles are described by files in \verb=styles/= directories. For example, a
file called \verb=styles/Foo.style= specifies the style named \verb=Foo=. By
convention styles start with a capital letter.

Each style file is in the format of a sequence of name-value pairs; one per
line. A list of all possible style file properties follows. Defaults are in
square brackets (you only have to specify the properties you want to override).

Generally there are no restrictions on any of the values (except that colour
properties have to be a valid colour name, and boolean properties must be
\verb=yes= or \verb=no= etc).
This means that strange things could happen if you pick
out-of-bounds combinations; for example you could specify a fold icon which is
wider than the fold margin it is supposed to fit in. If this occurs it should
be obvious visually and you can simply choose some more appropriate values.

The approach taken is to give you the power to change almost everything -- good
taste and spacing considerations are up to you.

If you create a style called \verb=Default.style=, the settings you override
with it will be used for any slides with no style, instead of the
hard-coded defaults.

\begin{verbatim}
textcolour=colour  [black]
bgcolour=colour    [white]
titlecolour=colour [black] (titlebar text)
barcolour=colour   [yellow] (set to bgcolour if you want it indistinguishable)
linkcolour=colour  [purple]

bullet1colour=colour [yellow]
bullet2colour=colour [cyan]
bullet3colour=colour [blue]
bullet1size=n [9] (specifies radius; shape is always circular)
bullet2size=n [7] (specifies radius; shape is always circular)
bullet3size=n [6] (always an ellipse, the lesser radius is two-thirds of n)
bullet1icon=path/to/picture.png [built-in circle]
bullet2icon=path/to/picture.png [built-in circle]
bullet3icon=path/to/picture.png [built-in ellipse]

foldcollapsedcolour=colour [green] (fixed shape)
foldexpandedcolour=colour [yellow] (fixed shape)
foldexposed1colour=colour [sky] (set to bgcolour if you want it invisible)
foldexposed2colour=colour [grey] (set to bgcolour if you want it invisible)
foldexposed3colour=colour [black] (set to bgcolour if you want it invisible)
foldcollapsedicon=path/to/picture.png [built-in]
foldexpandedicon=path/to/picture.png [built-in]

highlightcolour=colour [red] (applies to links and fold icons when clicked on)

bgimage=path/to/picture.png   [none] (this is always scaled to fit)
bgtexture=path/to/picture.png [none] (this is always tiled / cropped to fit)

bgbar=yes|no [yes] (bgimage / bgtexture also cover slide's title bar)
enablebar=yes|no [yes] (the slide title bar itself)
underlinelinks=yes|no [yes]

logo=x,y,path/to/picture.png [none] (you can repeat this for multiple logos)
  (x and y may be negative for placement relative to the right/bottom edges)

titlefont=font  [luxi/luxisb.ttf]
textfont=font   [luxi/luxisr.ttf]
fixedfont=font  [luxi/luximr.ttf]
boldfont=font   [luxi/luxisb.ttf]
italicfont=font [luxi/luxisri.ttf]

titlesize=n [40]
textsize=n  [36] (the size for plain text, bold and italic)
fixedsize=n [26]

linespacing=n  [42]
titlespacing=n [70]

bordercolour=colour [black]
pictureborder=n [1] (border width; 0 = off)
slideborder=n   [1] (border width; 0 = off)
barborder=n     [1] (this is the dividing line between the titlebar and
                     the main slide; 0 = off)

picturemargin=n [15]
topmargin=n     [0]
bottommargin=n  [20]
leftmargin=n    [15]
foldmargin=n    [30]
rightmargin=n   [30]

latexwidth=n [550]
latexscale=n [300]
latexbaselinestretch=n [85]
latexinclude=path/to/file.tex [none]
latexpreinclude=path/to/file.tex [none]
latexspaceabove=n [5]
latexspacebelow=n [0]

rulecolour=colour [black]
ruleheight=n [2]
rulewidth=percent [85]
rulespaceabove=n [10]
rulespacebelow=n [5]

headspaceabove=n [5]
headspacebelow=n [10]

# comment line in style files
\end{verbatim}

\section{Configuration files}

Multitalk searches for a configuration file in the following three
locations:
\begin{bulletlist}
\item \verb^/etc/multitalk.conf^
\item \verb^$MULTITALK_DIR/multitalk.conf^
\item \verb^~/.multitalk/multitalk.conf^
\end{bulletlist}

It will read any that exist. If an option is specified in more than
one file the locations further down the list take precedence.

There are currently only three config file options, and they are all
used to specify the location of the executables needed for the embedded
latex feature. This is useful if they cannot be found along the default
path. The options are:

\begin{verbatim}
latexcmd=path/to/latex     ["latex"]
dvipscmd=path/to/dvips     ["dvips"]
convertcmd=path/to/convert ["convert"]
\end{verbatim}

\section{File locations}

The Multitalk binary may be installed in any directory.
It locates its resources as follows.

When you invoke Multitalk you may supply a relative or absolute
pathname to the \verb=.talk= file. The directory containing
the \verb^.talk^ file is known as the \textit{project-directory}.
This is not necessarily the same as the current directory.

Multitalk searches for fonts in the following locations, in order
(stopping when it finds them):
\begin{numericlist}
\item \textit{project-directory}\verb^/fonts/^
\item \verb^~/.multitalk/fonts/^
\item \verb^$MULTITALK_DIR/fonts/^
\item \verb^/usr/local/share/multitalk/fonts/^
\end{numericlist}
It is not an error if any of those don't exist, or if the
\verb^MULTITALK_DIR^ environment variable hasn't been defined.

Multitalk looks for styles in all of these places:
\begin{numericlist}
\item \textit{project-directory}\verb^/styles/^
\item \verb^~/.multitalk/styles/^
\item \verb^$MULTITALK_DIR/styles/^
\item \verb^/usr/local/share/multitalk/styles/^
\end{numericlist}
If the same style appears in more than one of these directories, they
are \textbf{all} applied but settings from directories higher up this list
override those specified in the lower ones.

Files referred to in style files (e.g. the \verb=latexinclude= and
\verb=latexpreinclude= properties) are searched for in the same four
locations.

Images with absolute file names (starting with a \verb=/=) are looked up in
the usual way. If an image file name starts with \verb^~/^ or
\verb^~username/^ the tilde is expanded to the relevant user's home
directory.

Images with relative path names are searched for in the
following locations, in order (stopping once found):
\begin{numericlist}
\item \textit{project-directory}\verb^/^
\item \verb^~/.multitalk/gfx/^
\item \verb^$MULTITALK/gfx/^
\item \verb^/usr/local/share/multitalk/gfx/^
\end{numericlist}

\section{Screen size}

Multitalk's default display size is 1024x768 pixels, which currently
suits most video projectors well. If you need to change this, there
are two related sets of parameters you can adjust. Firstly, when invoking
Multitalk you may specify the following command-line option:

\verb^-displaysize=x,y^

Secondly, the .talk file may include the following directive:

\verb^@designsize=x,y^

The reason for this is that most talks are designed with a particular
screen size in mind, and will not appear correctly at a different
resolution (for example, some slides might not fit on the screen,
or unrelated parts of a talk might be visible). By including
\verb^designsize^ in your talk other users can see the intended
resolution. The \verb^displaysize^ option should instead be set to
the native resolution of the display viewing the talk. Both are
optional.

For example, if you are editing a presentation on a widescreen
laptop, but will be giving your talk using a 4:3 projector,
you might set \verb^designsize^ to 1024x768 and \verb^displaysize^
on the laptop to something like 1366x768 or 1024x600.

If \verb^displaysize^ and \verb^designsize^ differ, but have the
same aspect ratio, Multitalk opens its window or screen to
\verb^displaysize^, renders the talk normally, and then scales
it in the ratio \verb^displaysize^ : \verb^designsize^ so that it
fits the display. In this way the user sees exactly the same area
of the talk as they would at its original resolution.

Note that if Multitalk has to scale its output, the visual quality
will degrade (the algorithm used emphasises speed over quality);
also you will not gain any extra detail by scaling up to a larger size.

If \verb^displaysize^ and \verb^designsize^ differ and have a
different aspect ratio, Multitalk tries to keep the aspect ratio
of the slides unchanged, by picking the largest size of the
correct (design) aspect ratio which will fit on the display.
In windowed mode, it may therefore open a window which is smaller
than the specified \verb^displaysize^.

In fullscreen mode, Multitalk
relies on SDL's ability to centre and clip the output on the screen if
the requested resolution is not available. Typically, if the display
is wider than the design, you will see black bars to the left and
the right of the screen; if the display is taller than the design
there will be black bars at the top and the bottom of the screen.

Note that this will not always work if the display hardware is prepared
to actually sync to a non-native aspect ratio mode using hardware
scaling (rather than offering the next largest native mode). In this
case the aspect ratio will still be wrong, and quality will be
terrible due to the combination of both software and hardware scaling.
If this happens you may be forced to alter the talk's \verb^designsize^
and accept a different resolution.

\section{Images}

PNG, GIF and JPEG are supported.

You can have multiple images within each slide and transparency is supported,
so they do not have to be rectangular.

There are a few photos, diagrams, icons, logos, clipart and textures in the
\verb=gfx/= directory which are used by the examples. These are all freely
distributable, and are mostly taken from the open clipart project and eden
nature photos.

\section{Colours}

These 17 colour names are used in \verb^%^ directives within a talk and in
style files. They are all case-insensitive.

\begin{tabular}{llll}
\verb^White^       & \verb^Brown^    & \verb^Pink^      & \verb^LightGreen^\\
\verb^LightGrey^   & \verb^Red^      & \verb^Magenta^   & \verb^Green^\\
\verb^Grey^        & \verb^Orange^   & \verb^Purple^    & \verb^Cyan^\\
\verb^DarkGrey^    & \verb^Yellow^   &                  & \verb^Blue^\\
\verb^Black^       &                 &                  & \verb^Sky^\\
\end{tabular}

Colours can also be specified as a hash sign followed by 6
hexadecimal digits, e.g. \verb^#FFFF00^ for yellow. This works
both in \verb^%^ directives and style files.

\section{Fonts}

Multitalk ships with a range of freely-redistributable fonts in the fonts
directory, so that you can start using it ``out of the box''. You can also copy
your own Truetype fonts into this directory and use those as well.

Microsoft's free web fonts work particularly well, sadly due to license
restrictions we cannot include them in this package as \verb=.ttf= files
(they can
only be shipped in an archive format). You can however download the web fonts
for yourself from one of the various sites which redistribute them, and extract
them into the \verb=fonts/webfonts= directory.
The \verb=WebFonts.style= file shows how to
use them; uncomment the appropriate lines in this file to match the fonts. You
should make sure you end up with at least the files \verb=arial.ttf=,
\verb=arialbd.ttf=, \verb=ariali.ttf=, \verb=cour.ttf=, \verb=times.ttf=,
\verb=timesbd.ttf= and \verb=timesi.ttf=.

\section{Latex sections}

The use of latex syntax requires three external programs to be installed on
your machine: \verb=latex=, \verb=dvips= and \verb=convert=
(the latter is part of ImageMagick).
If these are not in your path you can specify their location using
the config file options \verb=latexcmd=, \verb=dvipscmd= and \verb=convertcmd=.

There can be any number of latex sections per slide and per talk. Each
one is processed independently by latex. Multitalk saves the output as
PNG files automatically in a directory called \verb=foo.latex/= (where your
talk file is \verb=foo.talk=). The PNG files have numeric names generated from
the latex source text. Next time you start Multitalk it won't run latex
again if the PNG output has already been generated, which improves
start-up time considerably and isolates you from the possibility of
something going wrong with latex on your presentation machine.

Incidentally, the \verb=.latex= directories are never cleaned up, so if you
make a lot of edits to your latex sections you may wish to delete the
PNG files within from time to time to save disk space (Multitalk will
regenerate the ones currently required when you run it). Otherwise
there is no need to pay any attention to the .latex directory.

A latex section may span multiple lines but it is not possible to
insert latex as a fraction of an ordinary line (i.e. a line break is
forced before and after each latex section). You can choose
left-aligned or centred latex and control the latex page width for word
wrap. The slide sizes in order to fit its latex contents as expected.

LaTeX sections begin and end with a line containing a single backslash,
like this:
\begin{verbatim}
\
Some latex $e=mc^2$
\
\end{verbatim}

If you want a LaTeX section to be centred then start it with a line
consisting of \verb^\)^ instead of \verb^\^, like this:
\begin{verbatim}
\)
Centred \fbox{latex}
\
\end{verbatim}

The main style file properties which relate to LaTeX sections are
\verb=latexwidth= and \verb=latexscale=. The latexwidth property specifies the
maximum width of the LaTeX page in screen pixels. The default is 550.
The latexscale property is a magnification factor which you can
use to match the LaTeX font size to that of your ordinary slide text.
Note that the default for this is 300.

The latex source text you include between the backslash lines should
not have a latex header or any \verb^\begin{document}^ or
\verb^\end{document}^,
because these are wrapped around the fragment for you by Multitalk. If
you need to change the document header you can set the
\verb=latexpreinclude= style property to the name of a file containing the
lines you wish to add. This file is inserted to all latex documents
just after the \verb^\usepackage^ lines in the header, so is a good place to
include any additional \verb^\usepackage^ declarations you need.

The \verb=latexinclude= style property is similar but the file specified
is included after the \verb=\begin= \verb={document}= line, immediately before
your source text fragment. This is a handy place to put latex body
commands which you need in all your latex segments, such as a change
of font, to avoid typing them repeatedly.

There is also a \verb=latexbaselinestretch= style property, which sets the
\verb^\baselinestretch^ variable in latex, but is expressed as an integer
percentage, so for example you write\\
\verb^latexbaselinestretch=120^\\
if you
want a stretch factor of 1.2. The default is actually 85 (0.85). This
is a convenience feature for quick matching of the latex and slide line
spacing, since you could also insert the command manually using the
\verb=latexinclude= property.

\section{Hints}

When you start Multitalk, the display will initially be centred on the
first slide in your talk (it's therefore a good idea if that is the
title slide).

When you add new slides to your talk, they will initially appear below and
slightly to the left of the location of the slide which preceeds them in the
\verb=.talK= file. This minimises the chance that slides are buried underneath each
other, even when you add multiple new slides at the same time. You will
probably want to zoom out immediately and drag them to your desired locations
with the mouse.

Inside a block of fixed-width text (delimited by \verb^$^ signs)
occurrences of \verb^*^ and
\verb^/^ are taken literally. If you want some bold or italic text inside a
fixed-width segment you therefore have to exit the block with \verb^$^ first.

Warning: all slides are currently pre-rendered, hence if you have a lot of
slides you need a lot of memory!

\section{HTML export}

The \verb^-export^ command-line option converts Multitalk presentations
to HTML. This is primarily intended for uploading a copy to a
website so that your audience may review the presentation after your
talk. Another benefit is that you can give your talk from the
HTML version as a last resort if you do not have a computer with
Multitalk installed on it (obviously this is not as satisfactory
as the real thing, since you cannot scroll around the web version
the same way).

When exporting \verb^foo.talk^, Multitalk will create a collection of
webpages in a sub-directory called \verb^foo.html^.
There is one page for each slide. Each page has a small navigation
image at the top (similar to the radar window in Multitalk), and the
slide image below.
The navigation image is an imagemap, so you can click on a slide's
outline there to load the appropriate page.
A cross on the navigation display marks the slide currently shown.

A title page called \verb^index.html^ is also generated, which shows one
large navigation image covering the whole talk.
The \verb^.talk^ filename and the title of the first slide are used as
a heading for the title page.
Clicking on the background of any navigation image (the area not
covered by any slide) always returns to the title page.

Hyperlinks (text and image) work as you would expect.
Card stacks are also implemented; these are rotated by clicking on
special links which appear between the navigation image and the
slide image if there is more than one card.
Folding sections are supported in a limited way; specifically two versions
of each slide are generated, one with all folding sections collapsed
and one with them all expanded. Clicking on any folding line switches
to the other representation, but it is not possible to selectively
expand certain sections on a given slide. This is because there could
be an exponential number of combinations, which would be wasteful
of disk space to store.

\section{What about handouts?}

It is not currently possible to automatically generate handouts from a
Multitalk presentation. This is partly due to the way it treats a talk
as an interactive experience, which is more effective because it is
less closely related to a paper document. It is more natural to record
a Multitalk presentation than to print it...

\section{Disclaimer}

Your attention is directed to the ``No Warranty'' section of the software
license. If Multitalk crashes during your presentation to a room full
of VIP's the author is not responsible!

That said, if you find a bug in Multitalk, please report it so it can
be fixed! Make sure you test your presentation thoroughly when you rehearse
it. I also recommend making use of the HTML export feature described
above, and keeping your presentation in both Multitalk and HTML
formats on a USB memory stick. This will allow you to show the
presentation in emergencies on any computer with a web browser,
even if it does not have Multitalk or an Internet connection.

\section{Contact}

Bug reports, feature requests and general feedback can be sent to:
\verb^dmi1000@cam.ac.uk^

\end{document}
